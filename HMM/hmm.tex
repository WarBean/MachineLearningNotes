\documentclass[11pt,a4paper]{article}
\usepackage{fontspec}
\usepackage[BoldFont, SlantFont, CJKnumber]{xeCJK}
\usepackage{amsmath}
\setCJKmainfont[BoldFont=Adobe Heiti Std R]{Adobe Song Std L}
\setCJKsansfont[BoldFont=Adobe Heiti Std R]{Adobe Kaiti Std R}
\setCJKmonofont{Adobe Fangsong Std R}
\XeTeXlinebreaklocale "zh" 
\XeTeXlinebreakskip = 0pt plus 1pt minus 0.1pt

\DeclareMathOperator*{\argmax}{arg\,max}

\begin{document} 

\section{最基本的HMM模型}
\subsection{模型定义}
\subsubsection{符号}

序列长度:$T$\\
一个状态序列:$s=s_1 s_2 ... s_T$\\
一个观察序列:$o=o_1 o_2 ... o_T$\\
状态符号数:$N$\\
观察符号数:$M$\\
状态符号集:$S=\{S_1, S_2, ... S_N\}$,有$s_t \in S, 1 \le t \le T$\\
观察符号集:$O=\{O_1, O_2, ... O_M\}$,有$o_t \in O, 1 \le t \le T$\\

\subsubsection{模型假设}

状态序列的一阶马尔可夫性,即每个状态变量仅仅依赖于上一个状态变量:
\begin{equation} P(s_t | s_1, s_2, ... s_{t - 1}) = P(s_t | s_{t - 1})\text{,其中$2 \le t \le T$} \end{equation}
时序平稳性,即转移概率分布不随时间变化:
\begin{equation} P(s_{t_1} | s_{t_1 - 1}) = P(s_{t_2} | s_{t_2 - 1})\text{,$2 \le t_1, t_2 \le T$} \end{equation}
每个观察变量仅仅依赖于与之对应的状态变量:
\begin{equation} P(o_t | o_1, o_2, ... o_{t - 1}, o_{t + 1}, ... o_T, s_1, s_2, ... s_T) = P(o_t | s_t) \end{equation}

\subsubsection{模型参数}

一般来说,状态符号和观察符号的个数$N$和$M$是由具体的问题场景预先确定下来了的,所以我们感兴趣的HMM模型参数如下:
$$ a_{ij} = P(s_t = S_j | s_{t - 1} = S_i) $$
$$ b_{jk} = P(o_t = O_k | s_t = S_j) $$
$$ \pi_i = P(s_1 = S_i) $$
其中$1 \le i, j \le N$,$1 \le k \le M$,$2 \le t \le T$,故所有的$a_{ij}$构成矩阵$A_{N \times N}$,所有的$b_{jk}$构成矩阵$B_{N \times M}$,所有的$\pi_i$构成向量$\pi_{N \times 1}$,记三元组:
$$ \lambda = (A, B, \pi) $$
就表示HMM模型的所有参数。

\subsection{三个基本问题}

当我们定义了如上所述的HMM模型之后,要用它来干一些有意义的事情之前,要先解决三个基本问题:
\begin{itemize}
\item 已知一组模型参数$\lambda = (A, B, \pi)$,给定一个观察序列一个观察序列$o=o_1 o_2 ... o_T$,如何\textbf{高效地}计算出现的概率$P(o | \lambda)$?
\item 已知一组模型参数$\lambda = (A, B, \pi_)$,给定一个观察序列一个观察序列$o=o_1 o_2 ... o_T$,如何按照某种有意义的准则来找出一个状态序列$s = s_1 s_2 ... s_T$,使之能够最好地“解释该观察序列的出现”?
\item 给定一个观察序列一个观察序列$o=o_1 o_2 ... o_T$,如何求使这个观察序列出现概率最大的一组模型参数$\lambda_0 = \argmax_{\lambda} P(o | \lambda) $?
\end{itemize}
第一个问题属于用模型进行评估(evaluation)的问题,第二个问题属于用模型和数据进行推断(inference)的问题\textbf{(我瞎猜的)},第三个问题属于对模型进行参数优化(parameter optimization)的问题。。

\subsubsection{第一个问题:Brute-Force算法}

现在,有了模型参数$\lambda$,有了观察序列$o=o_1 o_2 ... o_T$,我们可以先假设知道状态序列$s=s_1 s_2 ... s_T$,可以容易写出条件概率$P(o | s, \lambda)$和$P(s | \lambda)$,由这两个条件概率可以写出$P(o, s | \lambda)$:
\begin{equation}\begin{split}
P(o | s, \lambda)
& = P(o_1, o_2, ... o_T | s, \lambda)\\
& = \prod_{t = 1}^T P(o_t | s, \lambda) = \prod_{t = 1}^T P(o_t | s_t, \lambda)\\
& = \prod_{t = 1}^T B(s_t, o_t)
\end{split}\end{equation}
\begin{equation}\begin{split}
P(s | \lambda)
& = P(s_1, s_2, ... s_T | \lambda)\\
& = P(s_1 | \lambda) \prod_{t = 2}^T P(s_t | s_{t - 1}, \lambda)\\
& = \pi(s_1) \prod_{t = 2}^T A(s_{t - 1}, s_t)
\end{split}\end{equation}
\begin{equation}\begin{split}
P(o, s | \lambda)
& = P(o | s, \lambda) P(s | \lambda)\\
& = \pi(s_1) \prod_{t = 2}^T A(s_{t - 1}, s_t) \prod_{t = 1}^T B(s_t, o_t)\\
& = \pi(s_1) B(s_1, o_1) A(s_1, s_2) B(s_2, o_2) ... A(s_{T - 1}, s_T) B(s_T, o_T)
\end{split}\end{equation}
其中对于函数$A(\cdot, \cdot)$、$B(\cdot, \cdot)$、$\pi(\cdot)$,有:
$$ A(S_i, S_j) = a_{ij} $$
$$ B(S_j, O_k) = b_{jk} $$
$$ \pi(S_i) = a_{i} $$
事实上我们并不知道状态序列是什么,而每一种状态序列都有可能,因此需要对整个状态序列空间求和,把$s$"margin out"掉:
\begin{equation}\begin{split}
P(o | \lambda)
& = \sum_s P(o, s | \lambda)\\
& = \sum_{s_1, s_2, ... s_T} \pi(s_1) B(s_1, o_1) A(s_1, s_2) B(s_2, o_2) ... A(s_{T - 1}, s_T) B(s_T, o_T)
\end{split}\end{equation}
分析这个运算的复杂度:总共需要进行$O(N^T)$规模的求和,每个求和需要做$O(2T)$规模的乘积,总的复杂度是$O(2TN^T)$,这显然是不行的。

\subsubsection{第一个问题:sum-product算法}

观察式子(???),注意到它做了很多重复计算,例如$\pi(s_1)$跟$s_2$无关,却要在$s_2$上求和时重复地计算,更不要说后面的$s_3,s_4...$了,所以先想办法用乘法分配律先提出共同的因子。可以从$s_1$开始,从前到后:
\begin{equation}\begin{split}
P(o | \lambda)
& = \sum_{s_2, ... s_T} ... \sum_{s_1} \pi(s_1) B(s_1, o_1) A(s_1, s_2)\\
& = \sum_{s_3, ... s_T} ... \sum_{s_2} A(s_2, s_3) B(s_2, o_2) \sum_{s_1} A(s_1, s_2) B(s_1, o_1) \pi(s_1) \\
& = ...\\
& = \sum_{s_T} B(s_T, o_T) \sum_{s_{T - 1}} A(s_{T - 1}, s_T) B(s_{T - 1}, o_{T - 1}) ... \sum_{s_1} A(s_1, s_2) B(s_1, o_1) \pi(s_1) 
\end{split}\end{equation}
也可以从$s_T$开始,从后到前:
\begin{equation}\begin{split}
P(o | \lambda)
& = \sum_{s_1, ... s_{T - 1}} ... \sum_{s_T} A(s_{T - 1}, s_T) B(s_T, o_T)\\
& = \sum_{s_1, ... s_{T - 2}} ... \sum_{s_{T - 1}} A(s_{T - 2}, s_{T - 1}) B(s_{T - 1}, o_{T - 1}) \sum_{s_T} A(s_{T - 1}, s_T) B(s_T, o_T)\\
& = ...\\
& = \sum_{s_1} \pi(s_1) B(s_1, o_1) \sum_{s_2} A(s_1, s_2) B(s_2, o_2) ... \sum_{s_T} A(s_{T - 1}, s_T) B(s_T, o_T)
\end{split}\end{equation}








\end{document}