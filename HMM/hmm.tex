\documentclass[11pt,a4paper]{article}
\usepackage{fontspec}
\usepackage[BoldFont, SlantFont, CJKnumber]{xeCJK}
\setCJKmainfont[BoldFont=Adobe Heiti Std R]{Adobe Song Std L}
\setCJKsansfont[BoldFont=Adobe Heiti Std R]{Adobe Kaiti Std R}
\setCJKmonofont{Adobe Fangsong Std R}
\XeTeXlinebreaklocale "zh" 
\XeTeXlinebreakskip = 0pt plus 1pt minus 0.1pt

% 数学公式相关
\usepackage{amsmath, bm}
\DeclareMathOperator*{\argmax}{arg\,max}
\newcommand{\boldvec}[1]{\bm{#1}}
\newcommand*{\scale}[2][4]{\scalebox{#1}{$#2$}}
\newcommand*{\resize}[2]{\resizebox{#1}{!}{$#2$}}
\numberwithin{equation}{section}

% 首行缩进
\usepackage{indentfirst}
\setlength\parindent{2em}

\begin{document} 

\section{最基本的HMM模型}
\subsection{模型定义}
\subsubsection{符号}

\noindent
序列长度:$T$\\
一个状态序列:$s=s_1 s_2 ... s_T$\\
一个观察序列:$o=o_1 o_2 ... o_T$\\
状态符号数:$N$\\
观察符号数:$M$\\
状态符号集:$S=\{S_1, S_2, ... S_N\}$,有$s_t \in S, 1 \le t \le T$\\
观察符号集:$O=\{O_1, O_2, ... O_M\}$,有$o_t \in O, 1 \le t \le T$\\

\subsubsection{模型假设}

\noindent
状态序列的一阶马尔可夫性,即每个状态变量仅仅依赖于上一个状态变量:
\begin{equation} P(s_t | s_1, s_2, ... s_{t - 1}) = P(s_t | s_{t - 1})\text{,其中$2 \le t \le T$} \end{equation}
时序平稳性,即转移概率分布不随时间变化:
\begin{equation} P(s_{t_1} | s_{t_1 - 1}) = P(s_{t_2} | s_{t_2 - 1})\text{,$2 \le t_1, t_2 \le T$} \end{equation}
每个观察变量仅仅依赖于与之对应的状态变量:
\begin{equation} P(o_t | o_1, o_2, ... o_{t - 1}, o_{t + 1}, ... o_T, s_1, s_2, ... s_T) = P(o_t | s_t) \end{equation}

\subsubsection{模型参数}

一般来说,状态符号和观察符号的个数$N$和$M$是由具体的问题场景预先确定下来了的,所以我们感兴趣的HMM模型参数如下:
\begin{subequations}
\begin{align}
a_{ij} & = P(s_t = S_j | s_{t - 1} = S_i)\\
b_{jk} & = P(o_t = O_k | s_t = S_j)\\
\pi_i & = P(s_1 = S_i)
\end{align}
\end{subequations}
其中$1 \le i, j \le N$,$1 \le k \le M$,$2 \le t \le T$,故所有的$a_{ij}$构成矩阵$A_{N \times N}$,所有的$b_{jk}$构成矩阵$B_{N \times M}$,所有的$\pi_i$构成向量$\pi_{N \times 1}$,记三元组:
\begin{equation}
\lambda = (A, B, \pi)
\end{equation}
这就表示HMM模型的所有参数。

\subsection{三个基本问题}

当我们定义了如上所述的HMM模型之后,要用它来干一些有意义的事情之前,要先解决三个基本问题:
\begin{itemize}
\item 已知一组模型参数$\lambda = (A, B, \pi)$,给定一个观察序列一个观察序列$o=o_1 o_2 ... o_T$,如何\textbf{高效地}计算出现的概率$P(o | \lambda)$?
\item 已知一组模型参数$\lambda = (A, B, \pi_)$,给定一个观察序列一个观察序列$o=o_1 o_2 ... o_T$,如何按照某种有意义的准则来找出一个状态序列$s = s_1 s_2 ... s_T$,使之能够最好地“解释该观察序列的出现”?
\item 给定一个观察序列一个观察序列$o=o_1 o_2 ... o_T$,如何求使这个观察序列出现概率最大的一组模型参数$\lambda_0 = \argmax_{\lambda} P(o | \lambda) $?
\end{itemize}
第一个问题属于用模型进行评估(evaluation)的问题,第二个问题属于用模型和数据进行推断(inference)的问题\textbf{(我瞎猜的)},第三个问题属于对模型进行参数优化(parameter optimization)的问题。。

\subsubsection{第一个问题:暴力算法}

现在,有了模型参数$\lambda$,有了观察序列$o=o_1 o_2 ... o_T$,我们可以先假设知道状态序列$s=s_1 s_2 ... s_T$,可以容易写出条件概率$P(o | s, \lambda)$和$P(s | \lambda)$,由这两个条件概率可以写出$P(o, s | \lambda)$:
\begin{align}
\begin{split}
P(o | s, \lambda) 
& = P(o_1, o_2, ... o_T | s, \lambda) = \prod_{t = 1}^T P(o_t | s, \lambda) = \prod_{t = 1}^T P(o_t | s_t, \lambda)\\
& = \prod_{t = 1}^T B(s_t, o_t)
\end{split}
\\
\begin{split}
P(s | \lambda) \quad
& = P(s_1, s_2, ... s_T | \lambda) = P(s_1 | \lambda) \prod_{t = 2}^T P(s_t | s_{t - 1}, \lambda)\\
& = \pi(s_1) \prod_{t = 2}^T A(s_{t - 1}, s_t)
\end{split}
\\
\begin{split}
P(o, s | \lambda)
& = P(o | s, \lambda) P(s | \lambda) = \pi(s_1) \prod_{t = 2}^T A(s_{t - 1}, s_t) \prod_{t = 1}^T B(s_t, o_t)\\
& = \pi(s_1) B(s_1, o_1) A(s_1, s_2) B(s_2, o_2) ... A(s_{T - 1}, s_T) B(s_T, o_T)
\end{split}
\end{align}
其中对于函数$A(\cdot, \cdot)$、$B(\cdot, \cdot)$、$\pi(\cdot)$,有:
\begin{subequations}
\begin{align}
A(S_i, S_j) & = a_{ij}\\
B(S_j, O_k) & = b_{jk}\\
\pi(S_i) & = a_{i}
\end{align}
\end{subequations}
事实上我们并不知道状态序列是什么,而每一种状态序列都有可能,因此需要对整个状态序列空间求和,把$P(o, s | \lambda)$中的$s$"margin out"掉,即$P(o | \lambda) = \sum_s P(o, s| \lambda)$,展开后有:
\begin{equation}\label{P(o|lambda)}
P(o | \lambda) = \sum_{s_1, s_2, ... s_T} \pi(s_1) B(s_1, o_1) A(s_1, s_2) B(s_2, o_2) ... A(s_{T - 1}, s_T) B(s_T, o_T)
\end{equation}
分析这个运算的复杂度:总共需要进行$O(N^T)$规模的求和,每个求和需要做$O(T)$规模的乘积,总的复杂度是$O(TN^T)$,这显然是不行的。

\subsubsection{第一个问题:forward算法}

首先注意到,按照乘法分配律有:
\begin{equation}
\sum_{x = 1}^N f(x, y)g(y) = g(y) \sum_{x = 1}^N f(x, y)
\end{equation}

观察式子\eqref{P(o|lambda)}发现有类似的结构,同时注意到它做了很多重复计算,例如$\pi(s_1)$跟$s_2$无关,却要在$s_2$上求和时重复地计算,更不要说后面的$s_3,s_4...$了,所以先想办法用乘法分配律先提出共同的因子。先尝试从从$s_1$开始,从前到后:
\begin{equation}\label{forward_reduce}
\scalebox{0.87}{$P(o | \lambda) = \sum_{s_T} B(s_T, o_T) \sum_{s_{T - 1}} A(s_{T - 1}, s_T) B(s_{T - 1}, o_{T - 1}) ... \sum_{s_1} A(s_1, s_2) B(s_1, o_1) \pi(s_1)$}
\end{equation}

似乎复杂度降低了不少,但是这个式子在算法实现的角度上不好算,进一步观察发现它具有一定的周期递推特性,所以接下来通过引入一组新的函数$\alpha_t$来写出一个递归形式的表示:
\begin{equation}\label{alpha_t_definition}
\alpha_t(s_t) = 
	\begin{cases}
		B(s_1, o_1) \pi(s_1) & \text{当 $t = 1$}\\
		B(s_t, o_t) \sum_{s_{t - 1}} ... \sum_{s_{t - 2}} ... \pi(s_1) & \text{当 $2 \le t \le T$}
	\end{cases}
\end{equation}
其实就是把式子\eqref{forward_reduce}从后往前“截断”到$B(s_t, o_t)$左边的部分。按照定义,$\alpha_t(s_t)$具有如下递推性质:
\begin{equation}\label{alpha_t_recursive}
\alpha_t(s_t) = B(s_t, o_t) \sum_{s_{t - 1}} A(s_{t - 1}, s_t) \alpha_{t - 1}(s_{t - 1}) \text{,当 $2 \le t \le T$}
\end{equation}
这样从算法实现的角度就可以不断迭代从$\alpha_1(s_1)$一直迭代算到$\alpha_T(s_T)$,然后再最后跨一步算$p(o | \lambda)$:
\begin{equation}\label{alpha_t_final_shot}
p(o | \lambda) = \sum_{s_T} \alpha_T(s_T)
\end{equation}

考察上述计算的复杂度,$t = 1$时不用计算,从$t = 2 \rightarrow T$共$T - 1$次迭代,每次迭代按照递归公式\eqref{alpha_t_recursive}其实是这样的一个矩阵运算($\circ$表示阿玛达乘积Hadamard product,即逐位相乘):
\begin{equation}
\boldvec{\alpha_t} = A^T \cdot \boldvec{\alpha_{t - 1}} \circ \boldvec{b(o_t)}
\end{equation}
其中,列向量$\boldvec{\alpha_t}$、$\boldvec{b(o_t)}$分别为:
\begin{subequations}
\begin{align}
\boldvec{\alpha_t} & = [\alpha_t(S_1), \alpha_t(S_2), ... \alpha_t(S_N)]^T\\
\boldvec{b(o_t)} & = [B(S_1, o_t), B(S_2, o_t), ... B(S_N, o_t)]^T
\end{align}
\end{subequations}
故而每次迭代的复杂度是$O(N^2 + N)$,最后一次算$p(o | \lambda)$复杂度是$O(N)$,总的复杂度是$(T - 1) * O(N^2 + N) + O(N) = O(TN^2)$,比暴力算法的$O(TN^T)$高不知道哪去了

刚刚是把式\eqref{forward_reduce}从后往前“截断”到$B(s_t, o_t)$左边,但其实也可以“截断”到$B(s_t, o_t)$右边。定义一个新的函数$\alpha'_t$:
\begin{equation}\label{alpha'_t_definition}
\alpha'_t(s_t) = 
	\begin{cases}
		\pi(s_1) & \text{当 $t = 1$}\\
		\sum_{s_{t - 1}} ... \sum_{s_{t - 2}} ... \pi(s_1) & \text{当 $2 \le t \le T$}
	\end{cases}
\end{equation}
其递推性质为:
\begin{equation}\label{alpha'_t_recursive}
\alpha_t(s_t) = \sum_{s_{t - 1}} A(s_{t - 1}, s_t) B(s_{t - 1}, o_{t - 1}) \alpha_{t - 1}(s_{t - 1}) \text{,当 $2 \le t \le T$}
\end{equation}
从$\alpha_1(s_1)$一直迭代算到$\alpha_T(s_T)$,然后再最后跨一步算$p(o | \lambda)$:
\begin{equation}\label{alpha'_t_final_shot}
p(o | \lambda) = \sum_{s_T} B(s_T, o_t) \alpha_T(s_T)
\end{equation}
式\eqref{alpha'_t_definition}、\eqref{alpha'_t_recursive}、\eqref{alpha'_t_final_shot}与式\eqref{alpha_t_definition}、\eqref{alpha_t_recursive}、\eqref{alpha_t_final_shot}表示的计算过程是等价的,复杂度同样也是$O(TN^2)$。

推导到这里,可以问一个问题:$\alpha_t$和$\alpha'_t$是否具有某种概率意义?后面我们会发现,这个问题非常重要。

先从$\alpha_t$开始,观察其定义\eqref{alpha_t_definition},发现:
\begin{align}
\begin{split}\label{induction_init}
\alpha_1(s_1) & = B(s_1, o_1) \pi(s_1) \\
& = P(o_1 | s_1, \lambda) P(s_1 | \lambda) \\ 
& = P(o_1, s_1 | \lambda)
\end{split}
\\
\begin{split}
\alpha_2(s_2) & = B(s_2, o_2) \sum_{s_1} A(s_1, s_2) \alpha_1(s_1) \\
& = P(o_2 | s_2, \lambda) \sum_{s_1} P(s_2 | s_1, \lambda) P(o_1, s_1 | \lambda) \\
& = P(o_2 | s_2, \lambda) \sum_{s_1} P(o_1, s_1, s_2 | \lambda) \\
& = P(o_2 | s_2, \lambda) P(o_1, s_2 | \lambda) \\
& = P(o_1, o_2, s_2 | \lambda)
\end{split}
\\
\begin{split}
\alpha_3(s_3) & = \ldots \\
& = P(o_1, o_2, o_3, s_3 | \lambda)
\end{split}
\end{align}
于是猜想:
\begin{equation}\label{alpha_p_relation}
\alpha_t(s_t) = P(o_1, o_2, \ldots, o_t, s_t | \lambda) \quad 1 \le t \le T
\end{equation}
可用数学归纳法证明。假设$\alpha_t(s_{t - 1}) = P(o_1, o_2, \ldots, o_{t - 1}, s_{t - 1} | \lambda)$成立,则有:
\begin{equation*}
\begin{split}
\alpha_t(s_t) & = B(s_t, o_t) \sum_{s_{t - 1}} A(s_{t - 1}, s_{t - 1}) \alpha_{t - 1}(s_{t - 1}) \\
& = P(o_t | s_t, \lambda) \sum_{s_{t - 1}} P(s_t | s_{t - 1}, \lambda) P(o_1, o_2, \ldots, o_{t - 1}, s_{t - 1} | \lambda) \\
& = P(o_t | s_t, \lambda) \sum_{s_{t - 1}} P(o_1, o_2, \ldots, o_{t - 1}, s_{t - 1}, s_t | \lambda) \\
& = P(o_t | s_t, \lambda) P(o_1, o_2, \ldots, o_{t - 1}, s_t | \lambda) \\
& = P(o_1, o_2, \ldots, o_t, s_t | \lambda)
\end{split}
\end{equation*}
再结合初始条件\eqref{induction_init},可证明猜想\eqref{alpha_p_relation}。


\subsection{第一个问题:backward算法}
也可以从$s_T$开始,从后到前:
\begin{equation}\begin{split}
P(o | \lambda)
& = \sum_{s_1, ... s_{T - 1}} ... \sum_{s_T} A(s_{T - 1}, s_T) B(s_T, o_T)\\
& = \sum_{s_1, ... s_{T - 2}} ... \sum_{s_{T - 1}} A(s_{T - 2}, s_{T - 1}) B(s_{T - 1}, o_{T - 1}) \sum_{s_T} A(s_{T - 1}, s_T) B(s_T, o_T)\\
& = ...\\
& = \sum_{s_1} \pi(s_1) B(s_1, o_1) \sum_{s_2} A(s_1, s_2) B(s_2, o_2) ... \sum_{s_T} A(s_{T - 1}, s_T) B(s_T, o_T)
\end{split}\end{equation}







\end{document}