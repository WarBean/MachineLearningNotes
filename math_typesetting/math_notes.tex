\documentclass[11pt,a4paper]{article}
\usepackage{fontspec}
\usepackage[BoldFont, SlantFont, CJKnumber]{xeCJK}
\setCJKmainfont[BoldFont=Adobe Heiti Std R]{Adobe Song Std L}
\setCJKsansfont[BoldFont=Adobe Heiti Std R]{Adobe Kaiti Std R}
\setCJKmonofont{Adobe Fangsong Std R}
\XeTeXlinebreaklocale "zh" 
\XeTeXlinebreakskip = 0pt plus 1pt minus 0.1pt

% 数学公式相关
\usepackage{amsmath, bm}
\DeclareMathOperator*{\argmax}{arg\,max}
\newcommand{\boldvec}[1]{\bm{#1}}

\begin{document}

\section{杂项}

\subsection{如何插入代码}

可以用宏包verbatim插入代码,当然啦,也包括LaTeX源代码,这样就可以在pdf中显示LaTeX命令了。和数学公式类似,分为inline模式和display模式:

inline模式1:\verb!\verb|\section是一个LaTeX命令|!

inline模式2:\verb|\verb!\section是一个LaTeX命令!|

display模式:\\
\verb|\begin{verbatim}\section是一个LaTeX命令\end{verbatim}|

\section{数学公式}

\subsection{所谓的math mode}

有两种math mode,分别是夹杂在正常文字中间出现的in-line模式,和单独出现一行的stand-alone模式。\\
\TeX 风格的in-line模式:$ a + b = c $
\LaTeX 风格的in-line模式:\( a + b = c \)
\LaTeX 风格的in-line模式:\begin{math} a + b = c \end{math}
stand-alone模式:
stand-alone模式:
stand-alone模式:

\end{document}